\PassOptionsToPackage{unicode=true}{hyperref} % options for packages loaded elsewhere
\PassOptionsToPackage{hyphens}{url}
%
\documentclass[]{article}
\usepackage{lmodern}
\usepackage[ngerman]{babel}
\usepackage{amssymb,amsmath}
\usepackage{ifxetex,ifluatex}
\usepackage{fixltx2e} % provides \textsubscript
\ifnum 0\ifxetex 1\fi\ifluatex 1\fi=0 % if pdftex
  \usepackage[T1]{fontenc}
  \usepackage[utf8]{inputenc}
  \usepackage{textcomp} % provides euro and other symbols
\else % if luatex or xelatex
  \usepackage{unicode-math}
  \defaultfontfeatures{Ligatures=TeX,Scale=MatchLowercase}
\fi
% use upquote if available, for straight quotes in verbatim environments
\IfFileExists{upquote.sty}{\usepackage{upquote}}{}
% use microtype if available
\IfFileExists{microtype.sty}{%
\usepackage[]{microtype}
\UseMicrotypeSet[protrusion]{basicmath} % disable protrusion for tt fonts
}{}
\IfFileExists{parskip.sty}{%
\usepackage{parskip}
}{% else
\setlength{\parindent}{0pt}
\setlength{\parskip}{6pt plus 2pt minus 1pt}
}
\usepackage{hyperref}
\hypersetup{
            pdfborder={0 0 0},
            breaklinks=true}
\urlstyle{same}  % don't use monospace font for urls
\usepackage{color}
\usepackage{fancyvrb}
\usepackage{framed}
\newcommand{\VerbBar}{|}
\newcommand{\VERB}{\Verb[commandchars=\\\{\}]}
\DefineVerbatimEnvironment{Highlighting}{Verbatim}{commandchars=\\\{\}}
% Add ',fontsize=\small' for more characters per line
\newenvironment{Shaded}{}{}
\newcommand{\AlertTok}[1]{\textcolor[rgb]{1.00,0.00,0.00}{\textbf{#1}}}
\newcommand{\AnnotationTok}[1]{\textcolor[rgb]{0.38,0.63,0.69}{\textbf{\textit{#1}}}}
\newcommand{\AttributeTok}[1]{\textcolor[rgb]{0.49,0.56,0.16}{#1}}
\newcommand{\BaseNTok}[1]{\textcolor[rgb]{0.25,0.63,0.44}{#1}}
\newcommand{\BuiltInTok}[1]{#1}
\newcommand{\CharTok}[1]{\textcolor[rgb]{0.25,0.44,0.63}{#1}}
\newcommand{\CommentTok}[1]{\textcolor[rgb]{0.38,0.63,0.69}{\textit{#1}}}
\newcommand{\CommentVarTok}[1]{\textcolor[rgb]{0.38,0.63,0.69}{\textbf{\textit{#1}}}}
\newcommand{\ConstantTok}[1]{\textcolor[rgb]{0.53,0.00,0.00}{#1}}
\newcommand{\ControlFlowTok}[1]{\textcolor[rgb]{0.00,0.44,0.13}{\textbf{#1}}}
\newcommand{\DataTypeTok}[1]{\textcolor[rgb]{0.56,0.13,0.00}{#1}}
\newcommand{\DecValTok}[1]{\textcolor[rgb]{0.25,0.63,0.44}{#1}}
\newcommand{\DocumentationTok}[1]{\textcolor[rgb]{0.73,0.13,0.13}{\textit{#1}}}
\newcommand{\ErrorTok}[1]{\textcolor[rgb]{1.00,0.00,0.00}{\textbf{#1}}}
\newcommand{\ExtensionTok}[1]{#1}
\newcommand{\FloatTok}[1]{\textcolor[rgb]{0.25,0.63,0.44}{#1}}
\newcommand{\FunctionTok}[1]{\textcolor[rgb]{0.02,0.16,0.49}{#1}}
\newcommand{\ImportTok}[1]{#1}
\newcommand{\InformationTok}[1]{\textcolor[rgb]{0.38,0.63,0.69}{\textbf{\textit{#1}}}}
\newcommand{\KeywordTok}[1]{\textcolor[rgb]{0.00,0.44,0.13}{\textbf{#1}}}
\newcommand{\NormalTok}[1]{#1}
\newcommand{\OperatorTok}[1]{\textcolor[rgb]{0.40,0.40,0.40}{#1}}
\newcommand{\OtherTok}[1]{\textcolor[rgb]{0.00,0.44,0.13}{#1}}
\newcommand{\PreprocessorTok}[1]{\textcolor[rgb]{0.74,0.48,0.00}{#1}}
\newcommand{\RegionMarkerTok}[1]{#1}
\newcommand{\SpecialCharTok}[1]{\textcolor[rgb]{0.25,0.44,0.63}{#1}}
\newcommand{\SpecialStringTok}[1]{\textcolor[rgb]{0.73,0.40,0.53}{#1}}
\newcommand{\StringTok}[1]{\textcolor[rgb]{0.25,0.44,0.63}{#1}}
\newcommand{\VariableTok}[1]{\textcolor[rgb]{0.10,0.09,0.49}{#1}}
\newcommand{\VerbatimStringTok}[1]{\textcolor[rgb]{0.25,0.44,0.63}{#1}}
\newcommand{\WarningTok}[1]{\textcolor[rgb]{0.38,0.63,0.69}{\textbf{\textit{#1}}}}
\usepackage{longtable,booktabs}
% Fix footnotes in tables (requires footnote package)
\IfFileExists{footnote.sty}{\usepackage{footnote}\makesavenoteenv{longtable}}{}
\setlength{\emergencystretch}{3em}  % prevent overfull lines
\providecommand{\tightlist}{%
  \setlength{\itemsep}{0pt}\setlength{\parskip}{0pt}}
\setcounter{secnumdepth}{0}
% Redefines (sub)paragraphs to behave more like sections
\ifx\paragraph\undefined\else
\let\oldparagraph\paragraph
\renewcommand{\paragraph}[1]{\oldparagraph{#1}\mbox{}}
\fi
\ifx\subparagraph\undefined\else
\let\oldsubparagraph\subparagraph
\renewcommand{\subparagraph}[1]{\oldsubparagraph{#1}\mbox{}}
\fi

% set default figure placement to htbp
\makeatletter
\def\fps@figure{htbp}
\makeatother

\title{Dateisystem mit FUSE \\
  \large Dokumentation Betriebssysteme Labor \\
  \large Wintersemester 2018/2019}
\author{dosa1013, depa1016, odda1011}
\date{\today}

\begin{document}

\maketitle
\tableofcontents

\newpage

\hypertarget{aufgabenstellung-und-vorgaben}{%
\subsection{1. Aufgabenstellung und
Vorgaben}\label{aufgabenstellung-und-vorgaben}}

\hypertarget{aufgabenstellung}{%
\subsubsection{1.1 Aufgabenstellung}\label{aufgabenstellung}}

Die uns gestellte Aufgabe in diesem Labor bestand darin, ein Dateisystem
zu erstellen. Dieses soll verwendet werden, um einen Dateiträger zu
formatieren und somit Dateien mit den gewöhnlichen Attributen zu
unterstützen. Diese Attribute umschließen zum Beispiel den Namen, die
Größe, die Zugriffsrechte und die verschiedenen Zeitstempel der Dateien.
Weiterhin soll ermöglicht werden, dass ein Datenträger, der mit dem
Dateisystem formatiert wurde, in einem Verzeichnisbaum eingetragen wird.
Die Einbindung soll dann in einem freien, wählbarem und leerem
Verzeichnis erfolgen, in welchem der Inhalt des Datenträgers
anschließend erscheinen soll.

Anstatt wie bei traditionellen Dateisystemen mit Daten auf einem
Datenträger zu arbeiten, verwendeten wir eine Containerdatei. Außerdem
schreiben wir das Dateisystem nicht als Kernel Modul, sondern nutzen
FUSE. FUSE, was für ``File System In User Space'' steht, hilft einem
dabei, wie der Name schon sagt, Dateisysteme zu schreiben ohne dabei auf
Kernel Ebene programmieren zu müssen und wird dementsprechend vor allem
bei virtuellen Dateisystemen verwendet.

Während bei traditionellen Dateisystemen mit Kernel-Programmierung
Anfragen aus dem User Space den VFS (Virtual Filesystem Switch)
durchlaufen müssen, gefolgt vom Block-Layer, dem Input-Output-Layer, dem
Gerätetreiber bis schließlich der Datenträger erreicht wird, wird das
bei FUSE Dateisystemen vermieden.

Stattdessen werden Anfragen aus dem User Space vom VFS direkt an FUSE
weitergegeben, welches sich um die betroffenen Bereiche kümmert. FUSE
führt anschließend ein Programm aus, gibt diesem die Anfragen weiter und
erhält eine Antwort, die es zum anfragenden Programm weiterleitet.
Wodurch das virtuelle Dateisystem sich praktisch ebenfalls im User Space
befindet.

Im ersten Aufgabenteil war das Ziel mittels dem Kommando
\texttt{mkfs.myfs} eine Containerdatei zu erstellen, die alle nötigen
Strukturen enthält. Beim Erstellen sollen ausgewählte Dateien einmalig
in die Containerdatei kopiert werden. Nachdem diese durch FUSE in den
Verzeichnisbaum eingebunden wurde, soll es möglich sein Dateien lesen zu
können, aber noch nicht das Bearbeiten oder Löschen. Beim oben
beschriebenen Aufgabenteil war vor allem das Design des Dateisystems
entscheidend was den Aufbau und die Einteilung verschiedener Elemente
angeht. Anschließend ging es ans Erstellen und Befüllen des Datenträgers
durch das Kommando \texttt{mkfs.myfs}. Das Einbinden erfolgte über das
Kommando \texttt{mount.myfs}. Schlussendlich deckten wir mögliche
Fehlerquellen mit ausführlichen Testfällen ab.

Das Ziel des zweiten Aufgabenteils war es, nach dem Einbinden des
Datenträgers, Dateien bearbeiten und löschen zu können. Hierfür mussten
die FUSE-Operationen zum Anlegen, Ändern, Schreiben und Löschen von
Dateien implementiert werden.

Der dritte und letzte Teil der Laborabgabe ist das Anfertigen der
Dokumentation, über das von uns verfasste Programm.

\hypertarget{vorgaben}{%
\subsubsection{1.2 Vorgaben}\label{vorgaben}}

Das zu erstellende Dateisystem soll eine Größe von mindestens 30 MB
Platz für Dateien anbieten. Dementsprechend war die Größe aller
Verwaltungsstrukturen zu beachten um sicherzugehen, dass genug Platz für
Dateien übrig bleibt. Alle Dateien, die sich im Dateisystem befinden
sollen sich im Rootverzeichnis befinden und es soll keine weiteren
Verzeichnisse geben.

Als Konstanten wurden festgelegt: 
\begin{itemize}
  \tightlist
  \item
    Die maximale Länge eines Dateinamens \texttt{NAME\_LENGTH} auf 255 Charaktere. 
  \item 
    Die logische Blockgröße \texttt{BLOCK\_SIZE} mit 512 Byte. 
  \item 
    Die maximale Anzahl an Verzeichniseinträgen \texttt{NUM\_DIR\_ENTRIES} mit 64. 
  \item   
  - Die Anzahl der maximal geöffneten Dateien \texttt{NUM\_OPEN\_FILES} mit ebenfalls 64.
\end{itemize}

Die Übung soll von zwei bis vier Studenten durchgeführt werden.

Zur Bearbeitung des Projektes wurde ein Template bereitgestellt, welches
folgendes beinhaltet:

\begin{itemize}
\tightlist
\item
  Makefile
\item
  Testframework ``Catch''
\item
  diverse \texttt{.cpp} und \texttt{.h} Dateien, die nützliche
  Funktionalitäten und Vorlagen enthalten.
\end{itemize}

\hypertarget{read-only-filesystem}{%
\subsection{2. Read-Only Filesystem}\label{read-only-filesystem}}

\hypertarget{container-datei-aufbau}{%
\subsubsection{2.1 Container Datei
Aufbau}\label{container-datei-aufbau}}

Grundbestandteil des Dateisystems ist die Erstellung der Containerdatei
und diese anschließend zu befüllen. Die Containerdatei muss mindestens
eine Größe von 30 MB besitzen und wird in 512 Byte Blöcke
aufgeteilt. Maximal sollen bis zu 64 Dateien gespeichert werden können.
Als Gesamtgröße wählten wir 33.324.544 Bytes (32 MB) welche 65.536
Blöcke entsprechen. Das hat den Vorteil, dass wir über ein
\texttt{uint16\_t} (16 Bit Integer) alle Blöcke adressieren können.

Die Blöcke werden wie folgt genutzt:

\begin{longtable}[]{@{}llllll@{}}
\toprule
Name & Superblock & DMAP & FAT & Rootverzeichnis &
Dateien\tabularnewline
\midrule
\endhead
Größe & 1 & 128 & 256 & 64 & 65087\tabularnewline
Blockindex & 0 & 1-128 & 129-384 & 385-448 & 449-65535\tabularnewline
\bottomrule
\end{longtable}

\begin{itemize}
\item
  \textbf{Block 0} Enthält den Superblock, welcher nicht benutzt wird.
\item
  \textbf{Block 1-128} Enthält die DMAP in der abgelegt wird, welche
  Blöcke frei oder belegt sind. Dabei wird entweder der Charakter F für
  free, oder A für allocated an die Stelle des Datenblocks geschrieben.
  Um alle 65.536 Blöcke über ein Byte abzubilden, muss dieser Bereich
  128 Blöcke groß sein.
\item
  \textbf{Block 129-384} Enthält eine File Allocation Table (FAT) in der
  an der Stelle einer Blocknummer die Nummer des darauf folgenden Blocks
  abgelegt wird, welche die nächste Blocknummer beinhaltet. Beim letzten
  Block wird eine 0 geschrieben. Da pro Blocknummer werden 2 Byte
  benötigt werden, muss dieser Bereich 256 Blöcke groß sein.
\item
  \textbf{Block 385-448} Enthält das Rootverzeichnis mit Einträgen für
  jede im Dateisystem gespeicherte Datei. Diese Einträge beinhalten den
  Dateinamen, die Dateigröße, Benutzer/Gruppen-ID,
  Zugriffsberechtigungen, Zeitstempel für den letzten
  Zugriff/Veränderung/Statusänderung und den Zeiger auf den ersten
  Datenblock. Pro Datei wird hier ein Block benutzt. Da in unserem
  Dateisystem höchstens 64 Dateien Platz finden, wird dieser Bereich mit
  64 Blöcken bemessen.
\item
  \textbf{Block 449-65.535} Datenblöcke für Dateien. Abhängig von der
  Größe der Datei werden mehrere Blöcke verwendet.
\end{itemize}

\hypertarget{eigene-klassen-und-methoden}{%
\subsubsection{2.2 Eigene Klassen und
Methoden}\label{eigene-klassen-und-methoden}}

Für jeden der oben genannten Sektoren wird eine eigene Klasse
bereitgestellt. Diese Klassen haben Methoden für jegliche Interaktion
mit dem Sektor und werden in \texttt{mkfs.myfs.cpp} und
\texttt{myfs.cpp} verwendet.

\hypertarget{dmap-methoden}{%
\paragraph{2.2.1 DMAP-Methoden}\label{dmap-methoden}}

\texttt{void\ DMAP::create()}

Erstellt die DMAP, mit allen Bytes auf Free gesetzt.

\texttt{int\ DMAP::getFree(uint16\_t\ *pos)}

Kann verwendet werden, um einen leeren Block aus der DMAP zu bekommen.

\texttt{int\ DMAP::getFree(uint16\_t\ num,\ uint16\_t\ *arr)}

Über diese Methode können mehrere freie Blöcke auf einmal erhalten
werden.

\texttt{void\ DMAP::allocate(uint16\_t\ pos)}

Setzt einen bestimmten Block in der DMAP auf allocated.

\texttt{void\ DMAP::allocate(uint16\_t\ num,\ uint16\_t\ *arr)}

Setzt mehrere Blöcke auf allocated.

\hypertarget{fat-methoden}{%
\paragraph{2.2.2 FAT-Methoden}\label{fat-methoden}}

\texttt{uint16\_t\ FAT::read(uint16\_t\ curAddress)}

Liest die nächste Adresse aus der angegebenen Adresse.

\texttt{void\ FAT::write(uint16\_t\ curAddress,\ uint16\_t\ nextAddress)}

Schreibt die nächste Adresse an die angegebene Adresse.

\hypertarget{dpsfile-struct}{%
\paragraph{2.2.3 dpsFile-Struct}\label{dpsfile-struct}}

Um alle Informationen zu einer Datei zu speichern, wird ein eigenes
Struct verwendet, welches pro Datei im Rootverzeichnis gespeichert wird.

\begin{Shaded}
  \begin{framed}
\begin{Highlighting}[]
\KeywordTok{struct}\NormalTok{ dpsFile \{}
    \DataTypeTok{char}\NormalTok{ name[NAME_LENGTH]; }\CommentTok{// Name der Datei}
    \KeywordTok{struct}\NormalTok{ stat stat;       }\CommentTok{// Stat-Struct}
    \DataTypeTok{uint16_t}\NormalTok{ firstBlock;    }\CommentTok{// Addresse der ersten Datenblocks}
\NormalTok{\};}
\end{Highlighting}
\end{framed}
\end{Shaded}

\hypertarget{rootdir-methoden}{%
\paragraph{2.2.4 Rootdir-Methoden}\label{rootdir-methoden}}

\texttt{int\ RootDir::get(const\ char\ *name,\ dpsFile\ *fileData)}

Ruft die Dateiinformationen über den Dateinamen ab.

\texttt{int\ RootDir::exists(const\ char\ *name)}

Prüft, ob eine Datei im Dateisystem vorhanden ist.

\texttt{int\ RootDir::read(uint16\_t\ num,\ dpsFile\ *fileData)}

Liest Dateiinformationen aus dem Rootverzeichnis für die Datei mit der
Nummer \texttt{num}.

\texttt{int\ RootDir::write(dpsFile\ *fileData)}

Schreibt Dateiinformationen in das Rootverzeichnis. Wenn der gleiche
Name im Rootverzeichnis schon vorhanden ist, wird dieser überschrieben,
sonst wird ein neuer Eintrag angelegt.

\hypertarget{files-methoden}{%
\paragraph{2.2.5 Files-Methoden}\label{files-methoden}}

\texttt{int\ Files::read(uint16\_t\ *blocks,\ uint16\_t\ num,\ uint16\_t\ offset,\ char\ *buf)}

Liest eine Anzahl \texttt{num} von Blöcken aus der Containerdatei.

\texttt{int\ Files::write(uint16\_t\ *blocks,\ uint16\_t\ num,\ uint16\_t\ offset,\ size\_t\ size,\ const\ char\ *buf)}

Schreibt eine Anzahl von Blöcken in die Containerdatei.

\hypertarget{container-erstellung-und-befuxfcllung}{%
\subsubsection{2.3 Container Erstellung und
Befüllung}\label{container-erstellung-und-befuxfcllung}}

Die Containerdatei wird mit einem Kommando mit folgendem Aufbau erstellt: \texttt{mkfs.myfs\ containerfile\ {[}input-file\ ...{]}}. Die
einzelnen Dateien zum Befüllen werden an den Befehl angehängt. Wenn
keine Dateien angehängt werden, wird eine leere Containerdatei erzeugt.

\hypertarget{einbinden-und-lesen}{%
\subsubsection{2.4 Einbinden und Lesen}\label{einbinden-und-lesen}}

Das Einbinden des Dateisystems erfolgt durch das Kommando
\texttt{mount.myfs\ container\ logfile\ mountdir\ -s}. Der Parameter
\texttt{-s} steht für Single-Threaded-Mode, der verwendet werden soll, um
potentielle Fehler zu vermeiden, die durch Multithreading auftreten
können. Im Template wurde uns das Programm \texttt{mount.myfs}
übergeben, welches sicherstellt, dass das Kommando die richtige Syntax
befolgt und die benötigten Dateien die entsprechenden Voraussetzungen erfüllen. Sofern der
Zugriff auf den Container und die Logdatei erfolgreich waren und ein
Mountpoint existiert, kann die Initialisierung erfolgen.

\hypertarget{fuse-methoden}{%
\paragraph{2.4.1 FUSE-Methoden}\label{fuse-methoden}}

\texttt{*MyFS::fuseInit}

Öffnet die Logdatei und Containerdatei. Danach werden alle
Container-Objekte (DMAP, FAT, Root-Ordner und Dateien) erstellt.

\texttt{MyFS::fuseReaddir}

In der Methode muss für jeden Eintrag im Verzeichnis die übergebene
Funktion \texttt{filler} aufgerufen werden. Erst wird das für die
Einträge \texttt{.} und \texttt{..} getan. Diese stehen für das aktuelle
und das darüberliegende Verzeichnis. Dann wird über alle Dateien im
Rootberzeichnis iteriert und deren Namen an den filler übergeben.

\texttt{MyFS::fuseGetattr}

Befüllt statbuf mit Attributen aus dem Eintrag des Rootverzeichnis. Wenn
die angefragte Datei nicht existiert wird \texttt{-ENOENT}
zurückgegeben. Bei der Abfrage des Rootverzeichnisses selbst wird das
Stat-Struct mit den Werten:

\begin{Shaded}
  \begin{framed}
\begin{Highlighting}[]
\NormalTok{st_uid = getuid();        }\CommentTok{// UID des aktuellen Nutzers}
\NormalTok{st_gid = getgid();        }\CommentTok{// GID des aktuellen Nutzers}
\NormalTok{st_atime = time(NULL);    }\CommentTok{// alle Zeiten ...}
\NormalTok{st_mtime = time(NULL);    }\CommentTok{// auf den ...}
\NormalTok{st_ctime = time(NULL);    }\CommentTok{// aktuellen Zeitstempel}
\NormalTok{st_mode = S_IFDIR | }\DecValTok{0555}\NormalTok{; }\CommentTok{// Zugriffsberechtigungen auf 0555}
\NormalTok{st_nlink = }\DecValTok{2}\NormalTok{;             }\CommentTok{// Anzahl der Links auf 2}
\NormalTok{st_size = }\DecValTok{4096}\NormalTok{;           }\CommentTok{// Größe in Bytes auf 4096}
\NormalTok{st_blocks = }\DecValTok{8}\NormalTok{;            }\CommentTok{// Anzahl der 512 Byte Blöcke auf 8}
\end{Highlighting}
\end{framed}
\end{Shaded}

befüllt.

\texttt{MyFS::fuseOpen}

Öffnet eine Datei zum Lesen oder Schreiben. Beim Aufruf der Methode wird
die Anzahl der offenen Dateien um eins erhöht und das Structfeld
\texttt{fuse\_file\_info-\textgreater{}fh} wird mit dem Zeiger auf den
Ersten Datenblock der Datei beschrieben.

\texttt{MyFS::fuseRead}

Diese Methode wird immer direkt nach dem erfolgreichen fuseOpen
aufgerufen. Das übergebene Struct \texttt{fuse\_file\_info} beinhaltet
den vorher gespeicherten Filehandle. Nun können über die FAT alle
Blocknummern abgefragt und ausgelesen werden. Die gelesenen Bytes werden in den
Puffer \texttt{buf} kopiert und die Anzahl wird zurückgegeben.

\texttt{MyFS::fuseRelease}

Wird beim Schließen einer Datei aufgerufen. Die Anzahl der offenen
Dateien wird um eins reduziert und alle \texttt{toFile} Methoden der
Container-Objekte werden aufgerufen.

\hypertarget{read-and-write-filesystem}{%
\subsection{3 Read and Write
Filesystem}\label{read-and-write-filesystem}}

\hypertarget{erstellen-und-luxf6schen-von-dateien}{%
\subsubsection{3.1 Erstellen und Löschen von
Dateien}\label{erstellen-und-luxf6schen-von-dateien}}

\hypertarget{eigene-klassen-und-methoden-1}{%
\paragraph{3.1.1 Eigene Klassen und
Methoden}\label{eigene-klassen-und-methoden-1}}

Für das Löschen von Dateien mussten die Klassen \texttt{DMAP} und
\texttt{RootDir} um Methoden erweitert werden.

\texttt{void\ DMAP::setFree(uint16\_t\ pos)}

Setzt einen Block auf ``Free''.

\texttt{int\ RootDir::del(const\ char\ *name)}

Löscht den Eintrag aus dem Rootverzeichnis.


\hypertarget{fuse-methoden-1}{%
\paragraph{3.1.2 FUSE-Methoden}\label{fuse-methoden-1}}

\texttt{MyFS::fuseMknod}

Wird nur dann aufgerufen, wenn die Methode \texttt{MyFS::fuseGetattr} den
Fehlercode \texttt{-ENOENT} zurückgibt. Beim Erstellen einer neuen
Datei wird der gewünschte Name und die Zugriffsberechtigungen übergeben.
Die restlichen Filestats werden mit den Werten:

\begin{Shaded}
  \begin{framed}
\begin{Highlighting}[]
\NormalTok{st_blksize = }\DecValTok{512}\NormalTok{;}
\NormalTok{st_size = }\DecValTok{0}\NormalTok{;}
\NormalTok{st_blocks = }\DecValTok{0}\NormalTok{;}
\NormalTok{st_nlink = }\DecValTok{1}\NormalTok{;}
\NormalTok{st_atime = time(NULL);}
\NormalTok{st_mtime = time(NULL);}
\NormalTok{st_ctime = time(NULL);}
\NormalTok{st_uid = getgid();}
\NormalTok{st_gid = getuid();}
\end{Highlighting}
\end{framed}
\end{Shaded}

befüllt.

\texttt{MyFS::fuseWrite}

Schreibt vom \texttt{offset} bis \texttt{offset+size} den Inhalt aus dem
Puffer \texttt{buf} in eine Datei. Dabei wird zuerst ermittelt, ob neue
Blöcke für den Schreibvorgang nötig sind, oder ob die vorhandenen Blöcke
überschrieben werden. Wenn neue Blöcke verwendet werden, werden diese in
der FAT an die vorhandenen Blöcke angehängt. Nach dem Schreiben werden
die Filestats \texttt{st\_size} und \texttt{st\_blocks} auf die neue
Größe und \texttt{st\_mtime} und \texttt{st\_ctime} auf den aktuellen
Zeitstempel angepasst. Sollte der Schreibvorgang nicht möglich sein, weil
kein Platz im Dateisystem vorhanden ist, wird der Fehlercode
\texttt{-ENOSPC} zurückgegeben.

\texttt{MyFS::fuseUnlink}

Entfernt eine Datei, indem der Eintrag im Rootverzeichnis gelöscht wird
und die entsprechenden Datenblöcke in der DMAP freigegeben werden.

\hypertarget{testfuxe4lle}{%
\subsection{4. Testfälle}\label{testfuxe4lle}}

\hypertarget{testen-des-containers}{%
\subsubsection{4.1 Testen des Containers}\label{testen-des-containers}}

Das Testen des Erzeugens und
Befüllens der Containerdatei erfolgt ohne jegliche Abhängigkeit zu FUSE.
Dadurch konnte das Testen bereits parallel zum Schreiben des Codes für
die anderen Aufgabenstellungen erfolgen. Die Testcases sind mit dem
Testframework ``Catch'' geschrieben.

\hypertarget{test-dmap}{%
\paragraph{4.1.2 Test-DMAP}\label{test-dmap}}

In diesem Testcase werden die Methoden der DMAP-Klasse getestet. Erst werden einzelne Blöcke zurückgegeben und allokiert und dann
werden mehrere Blöcke aus der DMAP zurückgegeben und ebenfalls allokiert.

\hypertarget{test-fat}{%
\paragraph{4.1.1 Test-FAT}\label{test-fat}}

Diese Tests überprüfen, ob alle Lese- und Schreib-Methoden der Fat-Klasse
funktionieren.

\hypertarget{testen-der-fuse-operationen}{%
\subsubsection{4.2 Testen der Fuse
Operationen}\label{testen-der-fuse-operationen}}

Die Fuse Operationen wurden mit \texttt{Bash}-Scripten ausführlich
getestet. Dabei werden Linux-Commands ausgeführt und deren Exit-Code überprüft:

\begin{Shaded}
\begin{framed}
\begin{Highlighting}[]
\OperatorTok{<}\BuiltInTok{command}\OperatorTok{>}
\KeywordTok{if}\BuiltInTok{ [} \VariableTok{$?} \OtherTok{-eq}\NormalTok{ 0}\BuiltInTok{ ]}\NormalTok{; }\KeywordTok{then}
    \BuiltInTok{echo}\NormalTok{ -e }\StringTok{"command successful"}
\KeywordTok{else}
    \BuiltInTok{echo}\NormalTok{ -e }\StringTok{"command failed"}
\KeywordTok{fi}
\end{Highlighting}
\end{framed}
\end{Shaded}

\texttt{read.sh}

Das Skript erstellt eine Containerdatei und übergibt dabei eine
Testdatei. Dann wird die Containerdatei gemountet und die Datei über
\texttt{diff} mit dem Original verglichen.

\texttt{write.sh}

Mit diesem Script werden verschiedene Schreiboperationen getestet: -
Anlegen einer leeren Datei - Beschreiben der Datei - Text anhängen -
Datei überschreiben

\texttt{delete.sh}

Dieses Script testet die unlink Methode. Dazu wird eine Datei angelegt
und gleich wieder gelöscht.

\hypertarget{optimierungen}{%
\subsection{5. Optimierungen}\label{optimierungen}}

Beim ersten Programmieren der Container-Klassen haben alle Methoden
immer direkt die Daten aus der Containerdatei gelesen und geschrieben.
Dies hatte zufolge, dass Lese -und Schreiboperationen bei großen Dateien
lange andauerten. Um dies zu beschleunigen wurden für die Klassen
\texttt{DMAP}, \texttt{FAT} und \texttt{RootDir} jeweils ein Array
angelegt, welches im Konstruktor aus der Containerdatei befüllt wird und
durch den Aufruf der \texttt{toFile} Methode, in die Containerdatei
zurückgeschrieben wird.

Dadurch konnte die Zeit, die zum Schreiben von 30 MB gebraucht wird, um
das 42-fache reduziert werden.

\end{document}
